\newcommand{\trtitle}{Titel dieser Arbeit}

%%%%%%%%%%%%%%%%%%%%%%%%%%%%%%%%%%%%%%%%%%%%%%%%%%%%%%%%%%%%%%%%%%
\newcommand{\bmpstego}{stego}
%%%%%%%%%%%%%%%%%%%%%%%%%%%%%%%%%%%%%%%%%%%%%%%%%%%%%%%%%%%%%%%%%%


\documentclass[12pt,a4paper]{report}
%\documentclass[12pt,a4paper,onepage]{scrbook}

\usepackage[utf8]{inputenc}
\usepackage[T1]{fontenc}
\newcommand{\changefont}[3]{
\fontfamily{#1} \fontseries{#2} \fontshape{#3} \selectfont}

% Sprachen:
\usepackage[ngerman]{babel} % Silbentrennung Deutsch neue Rechtschreibung
\selectlanguage{ngerman}

\sloppy
%\usepackage{makeidx}
%\makeglossary
\makeindex

\usepackage{sty/abbreviations}

\usepackage{amsmath, marvosym} % Mathematik
\usepackage{times, url, geometry, amssymb, graphicx, booktabs}
\usepackage{fancyhdr} %Kopf- und Fußzeilen
\usepackage[colorlinks,pagebackref,pdfpagelabels]{hyperref} %Hyperlinks zw. Textstellen
\usepackage[hyphenbreaks]{breakurl}
\usepackage{color} % Farben
\hypersetup{
pdffitwindow=true,
pdfmenubar=true,
frenchlinks=false,
colorlinks=false,
bookmarksopen=true,
bookmarksnumbered=true,
pdfstartview=FitH,
pdftitle = {\trtitle},
pdfsubject = {\trtitle},
pdfauthor = {Robin Ruth},
pdfkeywords = {CODA, Bachelorarbeit, Ruth, SPA, Spring, UI-Design},
pdfcreator = {Adobe-Acrobat-Distiller},
pdfproducer = {LaTeX with hyperref and thumbpdf}
}
\usepackage{subfigure} % mehrere Abbildungen nebeneinander/übereinander
\usepackage{latexsym}

\geometry{a4paper,body={5.8in,9in}}
\setlength{\headheight}{15pt}

\usepackage{setspace} % 1,5 Zeilenabstand
\onehalfspacing
\setcounter{secnumdepth}{4}
\setcounter{tocdepth}{3} 

% Hurenkinder- und Schusterjungenregelung
\clubpenalty = 10000 % schliesst Schusterjungen aus
\widowpenalty = 10000 % schliesst Hurenkinder aus

% aller Bilder werden im Unterverzeichnis figures gesucht:
\graphicspath{{figures/}}

% Headers:
%\pagestyle{headings}
\pagestyle{fancy}
\pagestyle{headings}

% Literaturverzeichnis
\usepackage{bibgerm}
%\usepackage{natbib}
\bibliographystyle{gerunsrt} % Literaturangaben nach Auftreten sortieren %{gerplain}

\usepackage{listings} % f�r Formatierung in Quelltexten
\definecolor{grau}{gray}{0.25}
\lstset{
	extendedchars=true,
	basicstyle=\scriptsize\ttfamily,
	%basicstyle=\tiny\ttfamily,
	tabsize=2,
	keywordstyle=\textbf,
	commentstyle=\color{grau},
	stringstyle=\textit,
	numbers=left,
	numberstyle=\tiny,
	% f�r sch�nen Zeilenumbruch
	breakautoindent  = true,
	breakindent      = 2em,
	breaklines       = true,
	postbreak        = ,
	%prebreak         = \raisebox{-.8ex}[0ex][0ex]{\ensuremath{\lrcorner}},
	prebreak         = \raisebox{-.8ex}[0ex][0ex]{\Righttorque},
}

