\chapter{Introduction}

Artificial intelligence (AI) has in recent years become one of those hype words, evoking everything from sceneries of certain doom at the hand of uncontrollable robots to visions of plenty, where ai puts us on a path to enlightenment. Much more quietly if perhaps not quite as grand AI has indeed changed how we live our lives, recommending songs to hear and items to buy, steering cars and investing money, making homes listen to their inhabitants and beating world champions in their chosen field.\\
As a branch of AI, machine learning (ML) has, aided by the ongoing digitalization and the generation of big data, become a strong tool in the toolkit of modern data analyzation and software development. It allows a completely new level of automation, having a program not only follow a set piece of rules to their conclusion but to create those rules in the first place. \\
With supervised learning, programs may now find underlying rules behind large data collections and use it to label new input. Is an email spam, is a patient ill and how much is your car worth? Questions that, given the right information and training a program can now predict with high accuracy.\\
That is the point where automation today comes to its end. Finding a fitting ML algorithm and configuring it to resolve a problem is a task that in most cases must be done by ML experts. And just as diverse as the problems are that may be solved with ML and supervised learning are the ways to configure these algorithms, making this a nontrivial challenge. \\
To automate this process, GT-ARC (German Turkish Advanced Research Center for ICT) has started CODA, a fundamental research project in algorithm selection and hyperparameter optimization. \cite{gt-arc}  \\
CODA, however, does not come with an easy to use visual user interface (UI). With Sparked an interface to the CODA project will be created, allowing ML enthusiasts and specialists to use the developed solutions and giving the team an interface to demonstrate CODAs capabilities. 
