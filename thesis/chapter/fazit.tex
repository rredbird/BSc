\chapter{Conclusion and future development}
Creating Sparked became an interesting experience in knowing technology versus having mastered it. Being able to check off a list of technologies and coding paradigms that should go into a web application undoubtedly helps to know what needs to be developed. But having the actual knowledge to do so is a different thing altogether and getting stuck on simple things like dependency injection with an unknown framework or learning the ins and outs of how a specific logger has to be configured is both tedious and very time-consuming. 

The Sparked backend does what it is supposed to do. It is created with modern but tried technologies, the code is reasonably clean and should be simple enough to extend. On the front side of the program, this becomes more complicated. Angular offers a lot of tools to structure its code for reuse, which has only been used to a small degree. Here the code quality would have benefited even more from a real frontend developer with Angular experience. 

That said, the frontend connects to the backend to collect the necessary data offers and offers the functionality the requirements asked. The UI has been designed in a way to be simple to use and good to read. 

As can be seen in the previous chapter, the functional requirements have in principle been met. But the evaluation was quite shallow, only testing with a small handful of test Orders. Showing only the results for the target metric, too, should be expanded on. 

Even more problematic is the evaluation of the non-functional requirements. The visual guidelines have in principle be followed, but a qualitative evaluation of the UI was not done.
There are a couple of areas, where future development could help Sparked become a more useful application.

The UI would probably benefit a lot from a domain expert going over the different views and deciding what information the user would need. It is often the simple things, like what fields are needed to quickly decide if a user would want to open an Order out of the 20 other visible Orders, that give a big benefit to the usability. All data should be there right under the hood and having that knowledge should allow for easy changes.

Simple search capabilities on all lists would make a marked improvement. Even now there are lists that are long enough to warrant searching, with the continued development of CODA and usage of Sparked this will become more pressing as list become longer.

Last but not least the evaluation page should be extended. Since all relevant data is now available in the evaluation workflow, it should be possible to extend the evaluation page without changing anything else, reducing the amount of work needed to a minimum. 

All in all, while not a full success, the project can be considered a good stepping stone towards to create a UI that can adequately represent such a powerful tool as CODA.