\chapter{Conclusion and future development}
Creating Sparked became an interesting experiment in the difference between knowing a technology versus having mastered it. Being able to check of a list of technologies and coding paradigms that should go into a web application undoubtedly helps to know what needs to be developed, but having the actual knowledge how specific libraries work is a different thing altogether and getting stuck on simple things like dependency injection with an unknown framework or learning the ins and outs of how a specific logger has to be configured is both tedious and very time consuming. \\\\
Looking at the program, the Sparked backend does what it is supposed to do. It is created with modern but tried technologies, the code is reasonably clean and should be simple enough to extend. On the front side of the program this becomes more complicated. Angular offers a lot of tools to structure its code for reuse, which have only been used to a small degree. Here the code quality would have benefited from a real frontend developer with Angular experience. \\
That said, the frontend works, it connects to the backend to collect the necessary data offers and offers the functionality the requirements asked. The UI has been designed in a way to simple, easy to use and good to read. \\
But while functionality can easily be tested by creating and starting an order and once finished looking at its evaluation page, assessing the programs look and feel can not be done by a single person, as it is rather subjective. Because of time constraints a qualitative evaluation of the user interface was not possible, the readers must make their own assessment.\\\\

In the future the UI would probably benefit a lot from a domain expert going over the different views, deciding what information the user will need on each of them. It is often the simple things, like what fields are needed to quickly decide if a user would want to open an order out of the 20 other visible orders, that give a big benefit to the usability. All data should be there right under the hood and having that domain knowledge should allow for an easy change.\\
Next to this, simple search capabilities on all lists would make a marked improvement. Even now there are lists that are long enough to warrant searching, with continued development of CODA and usage of Sparked this will become more pressing as list become longer. \\\\

All in all while, not a full success, the project can be considered a good stepping stone towards to create a UI that can adequately represent such a powerful tool as CODA.
