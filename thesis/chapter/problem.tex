\chapter{Problemanalyse}
Länge: ca. 5 - 15 Seiten\\\\
Das Kapitel Problemanalyse dient dazu, das in der Einleitung identifizierte und eingegrenzte Problem auf seine Ursachen zurückzuführen und so Lösungsmöglichkeiten zu entwickeln. Hierdurch wird die Problembezogenheit der entwickelten Lösung sichergestellt. Wenn möglich, ist durch eine Literaturrecherche nachzuweisen, dass bisher keine geeigneten Lösungen existieren. Der Betreuer hilft bei der Entscheidung, ob die Problemanalyse ausreichend tief erfolgt ist. Häufig führt eine hinreichend genaue Problemanalyse zu präziseren und damit kürzeren Lösungskonzepten.\\\\

\noindent Kapitelzusammenfassung am Ende:\\
Der Übergang von der Problemanalyse zur Konzeptentwicklung stellt eine wichtige Nahtstelle innerhalb der Arbeit dar, da von einer betrachtend-analysierenden Perspektive auf eine konstruktiv-kreative Perspektive gewechselt wird. Daher empfiehlt es sich, an dieser Stelle die Ergebnisse der Problemanalyse zusammenzufassen.

