\chapter{Problem analysis}
Länge: ca. 5 - 15 Seiten\\\\
Das Kapitel Problemanalyse dient dazu, das in der Einleitung identifizierte und eingegrenzte Problem auf seine Ursachen zurückzuführen und so Lösungsmöglichkeiten zu entwickeln. Hierdurch wird die Problembezogenheit der entwickelten Lösung sichergestellt. Wenn möglich, ist durch eine Literaturrecherche nachzuweisen, dass bisher keine geeigneten Lösungen existieren. Der Betreuer hilft bei der Entscheidung, ob die Problemanalyse ausreichend tief erfolgt ist. Häufig führt eine hinreichend genaue Problemanalyse zu präziseren und damit kürzeren Lösungskonzepten.\\\\

\noindent Kapitelzusammenfassung am Ende:\\
Der Übergang von der Problemanalyse zur Konzeptentwicklung stellt eine wichtige Nahtstelle innerhalb der Arbeit dar, da von einer betrachtend-analysierenden Perspektive auf eine konstruktiv-kreative Perspektive gewechselt wird. Daher empfiehlt es sich, an dieser Stelle die Ergebnisse der Problemanalyse zusammenzufassen.


- Changes in the requirements make a redevelopment necessary. This is impractical and to be avoided in the future. The usage of well known technologies will allow more people to easily work on this project. Modularisation will help to allow changes at one side, without the need to propagate those changes throughout the entire application.

\section{A website to present CODAs capabilities}
- This is not a classical web application but a website that is mainly used for presentations. Visual components are very important. Making everything look nice has a very high priority.

- Should help evaluate how good a selected optimization was. 

\section{Facilitate change}
- Changes in CODA backend should propagate to the ui without changes to the code. 
- Changes to the code should be made as easy as possible.
- Wants to showcase CODA, but CODA will change.

\section{questions}
\begin{itemize}
    \item Focus on showing it in a presentation or allowing access as a presentation object site
    \item Do we expect random traffic of people who have not been introduced to CODA?
    \item Should there be a an introduction to the CODA project on the page?
    \item Should the basic concepts be described or can we assume professionals only?
    \item Is this at all wanted for productiv work or just for presentations? (Kickoff requrements say demo system, not for every day use.)
\end{itemize}

\section{Summary}
Create a demosystem that allows to showcase CODAs capabilities. It should make it easy to use in presentations and be designed to be flexible 